

\documentclass{beamer}
\usetheme{CambridgeUS}
\mode<presentation>
{
\setbeamercovered{transparent}
}
\usepackage{amsmath,amsfonts,amssymb}
\usepackage{german}
\usepackage[center]{caption}
\usepackage[utf8]{inputenc}
\usepackage[ngerman]{babel}
\usepackage{graphicx}
\usepackage{svg}
\usepackage{pdfpages}
\usepackage{placeins}
\usepackage[decimalsymbol=comma]{siunitx}


% font definitions, try \usepackage{ae} instead of the following
% three lines if you don't like this look
\usepackage{mathptmx}
\usepackage[scaled=.90]{helvet}
\usepackage{courier}
\beamertemplatenavigationsymbolsempty 
\setbeamertemplate{footline}{}

%\usepackage[T1]{fontenc}
\title[Bachelorarbeit]{Untersuchung des Wachstums von Au auf Re(0001) mittels LEED und STM sowie
Aufbau und Test einer Sprühdepositionsapparatur}
\author[V. Grimm]{Verena Grimm}


\institute[]{
Vortrag zur Bachelorarbeit in Physik\\
Fachbereich Physik, Mathematik und Informatik (FB 08)\\
Johannes Gutenberg-Universität Mainz
}

\date{00.00.2014}


% Logo Universität
\titlegraphic{\includegraphics[width=4cm]{bilder/JGU-Logo_sw_high.jpg}\hspace*{-8.5cm}}
% Logo auf jeder Seite rechts unten:
% \pgfdeclareimage[height=1.5cm]{university-logo}{bilder/JGU-Logo_sw_high.jpg}
%  \logo{\pgfuseimage{university-logo}}



\begin{document}

\begin{frame}
\titlepage
\end{frame}

\begin{frame}
\frametitle{Inhalt}
\tableofcontents
% You might wish to add the option [pausesections]
\end{frame}


\begin{frame}
\frametitle{Warum überhaupt?}
%\framesubtitle{Subtitles are optional}
\begin{itemize}\setlength{\itemsep}{+15pt}
  \item   
  \item
\end{itemize}
\end{frame}



%___________________________________________________________________________________________________________________________________________
%___________________________________________________________________________________________________________________________________________
%___________________________________________________________________________________________________________________________________________



\section{Wachstum von Au auf Re(0001)}

%___________________________________________________________________________________________________________________________________________
%___________________________________________________________________________________________________________________________________________


\begin{frame}
\frametitle{}
%\framesubtitle{Subtitles are optional}
\begin{center}
\textcolor{darkred}{\huge{1. Teil:\\ \vspace{0.5cm}
Untersuchung des Wachstums von Au \\ \vspace{0.5cm} auf Re(0001)}}
\end{center}
\end{frame}

%___________________________________________________________________________________________________________________________________________
%___________________________________________________________________________________________________________________________________________


\subsection[Motivation]{Motivation}

\begin{frame}
\frametitle{Warum Gold? Warum Rhenium?}
%\framesubtitle{Subtitles are optional}
\begin{itemize}\setlength{\itemsep}{+15pt}
 	\item Gold inert, d.h. wenig Reaktion mit Restgasmolekülen im Ultrahochvakuum
 	\item gute Adsorption organischer Moleküle auf Goldoberflächen
 \end{itemize}
 \vspace{1cm}
 \begin{itemize}\setlength{\itemsep}{+15pt}
  	\item Rhenium: hoher Schmelzpunkt ($3186^\circ$C)
  	\item bleibt nach Verarbeitung (Schweißen, Schmieden, \ldots) duktil
 \end{itemize}
\end{frame}

%___________________________________________________________________________________________________________________________________________
%___________________________________________________________________________________________________________________________________________


\subsection[Grundlagen]{Grundlagen: STM und LEED}

\begin{frame}
\frametitle{LEED (Low Energy Electron Diffraction)}
\framesubtitle{Beugung niederenergetischer Elektronen an Oberflächen}
\begin{figure}[htbp]
	\begin{minipage}[b]{0.45\textwidth}
		\includegraphics[]{bilder/leedkleiner}
	\end{minipage}
	\hspace{0.4cm}
	\begin{minipage}[b]{0.45\textwidth}
		\begin{itemize}
 		\item Elektronenstrahl durch Glühemission mit \SI{20}{eV} bis \SI{400}{eV}
 	 	\item niedrige Eindringtiefe der Elektronen
 	 	\item Fokussierung und "`Sortierung"' durch Gitter vor Schirm
 	 	\item betrachte elastisch rückgestreute Elektronen auf fluoreszierendem Schirm
		\end{itemize}
	\end{minipage}
\end{figure}
\end{frame}

\begin{frame}
\frametitle{LEED (Low Energy Electron Diffraction)}
\begin{figure}[H]
\begin{minipage}[b]{0.45\textwidth}
		\begin{itemize}
 		\item Betrachte elastische Streuung an oberster Schicht mit $k=k'$
 	 	\item konstruktive Interferenz allgemein, wenn $\vec{k}-\vec{k'}=\Delta \vec{k}=\vec{G}$~~~und
 	 	$\vec{a}_i\cdot \Delta \vec{k}=2\pi h_i,~~~~~~~i=1, 2, 3$
 	 	\item bei Oberflächenstreuung: $\Delta\vec{k}_{||}=\vec{k}_{||}-\vec{k'}_{||}\overset{!}{=} \vec{G}$
 	 	\item in Ewald-Konstruktion werden Punkten $(h_1, h_2)$ Linien $h_3$ zugeordnet
 	 	\item Schnittstellen Linien-Kugel: Interferenz!
		\end{itemize}
	\end{minipage}
	\hspace{0.4cm}
	\begin{minipage}[b]{0.45\textwidth}
		\includesvg[svgpath=bilder/]{ewald}
	\end{minipage}
\end{figure}
\end{frame}




\begin{frame}
\frametitle{STM (Scanning Tunneling Microscope)}
\framesubtitle{Rastertunnelmikroskopie}
\begin{itemize}\setlength{\itemsep}{+15pt}
  \item 3D Aufnahme von Oberflächen elektrisch leitender Festkörper
  \item Annäherung leitender Spitze an Oberfläche bis auf wenige {\AA} $\Rightarrow$ mit angelegter
  Spannung fließt messbarer Strom
  \item Mechanismus beruht auf Tunneleffekt
  \item Rastern der Oberfläche mit Spitze bei Messung des Stroms: topographisches Bild der
  Oberfläche mit atomarer Auflösung
\end{itemize}
\end{frame}

\begin{frame}
\frametitle{STM (Scanning Tunneling Microscope)}
% \begin{figure}[H]
% 	\begin{minipage}[b]{0.45\textwidth}
% 		\vspace{-3cm}
% 		\begin{itemize}\setlength{\itemsep}{+15pt}
% 		  \item Berechnungen schwer durch i.A. unbekannte Wellenfunktion der Spitze
% 		  \item Ausdruck für Strom nur durch Näherungen erreichbar
% 		  \item $I\propto U \rho_{s}(E_F) \rho_P(\vec{r}_0, E)e^{-2\kappa d}$
% 		\end{itemize}
% 	\end{minipage}
% 	\hspace{0.2cm}
% 	\begin{minipage}[b]{0.45\textwidth}
% 		\includesvg[svgpath=bilder/]{spitze}
% 	\end{minipage}
	 \begin{columns}
\column{.45\textwidth}
\begin{figure}[H]
\begin{center}
\begin{itemize}\setlength{\itemsep}{+15pt}
		  \item Berechnungen schwer durch i.A. unbekannte Wellenfunktion der Spitze
		  \item Ausdruck für Strom nur durch Näherungen erreichbar
		  \item $I\propto U \rho_{s}(E_F) \rho_P(\vec{r}_0, E)e^{-2\kappa d}$
		\end{itemize}
\end{center}
\end{figure}

   \column{.45\textwidth}
\begin{figure}[H]
\begin{center}
\includesvg[svgpath=bilder/]{spitze}
\end{center}
\end{figure}
\end{columns}
\end{frame}

%___________________________________________________________________________________________________________________________________________
%___________________________________________________________________________________________________________________________________________


\subsection[Versuchsaufbau]{Versuchsaufbau}

\begin{frame}
\frametitle{Versuchsaufbau}
\begin{itemize}\setlength{\itemsep}{+15pt}
  \item UHV-Apparatur mit \SI{e-10}{mbar}
  \item Flashen, Tempern, Bedampfen der Probe mit Gold
  \item Bestimmung der aufgedampften Menge über Schwingquarz
  \item LEED, STM
\end{itemize}
\end{frame}

\begin{frame}
\frametitle{Aufbau der UHV-Apparatur}
\begin{figure}[H]
\centering
\sffamily
\includesvg[svgpath=bilder/]{uhv-apparatur}
\end{figure}
\end{frame}

\begin{frame}
\frametitle{Das STM}
\begin{figure}[H]
\centering
\sffamily
\includesvg[svgpath=bilder/]{stm-tisch}
\end{figure}
\end{frame}

%___________________________________________________________________________________________________________________________________________
%___________________________________________________________________________________________________________________________________________


\subsection[Ergebnisse]{LEED}

\begin{frame}
\frametitle{LEED des Re-Kristalls}
\begin{figure}[htbp]
	\begin{minipage}[b]{0.45\textwidth}
	\includegraphics[width=\textwidth]{bilder/unbedampft_E207}
	\caption*{Rand des Kristalls}
	\end{minipage}
	\hspace{0.2cm}
	\begin{minipage}[b]{0.45\textwidth}
	\includegraphics[width=\textwidth]{bilder/unbedampft_E207_MitteKristall.jpg}
	\caption*{Mitte des Kristalls}
	\end{minipage}
\end{figure}
\end{frame}

\begin{frame}
\frametitle{LEED des Re-Kristalls}
\vspace{-0.2cm}
\begin{figure}[H]
\begin{minipage}[b]{0.45\textwidth}
\includesvg[svgpath=bilder/]{unbedampft_E207-winkel}
\end{minipage}
\begin{minipage}[b]{0.45\textwidth}
\includesvg[svgpath=bilder/]{bcc-hcp}
\end{minipage}
\caption*{a) Eingezeichnete größt- und
kleinstmögliche Winkel.\\ b) (0001)-Oberfläche der hcp-Struktur, $\alpha=120^{\circ}$ und
$\beta=60^{\circ}$.\\ c) (110)-Oberfläche der bcc-Struktur, $\alpha\approx109{,}5^{\circ}$ und
$\beta\approx70{,}5^{\circ}$.}
\end{figure}
\vspace{-0.5cm}
\small\begin{table}[H]
\centering
\begin{tabular}{ c  r  r  r}
Winkel &	min	 &	max & Mittel \\
 \hline                       
 $\alpha$&105,8$^{\circ}$&111,4$^{\circ}$&108,6$^{\circ}$\\
 $\beta$&71,5$^{\circ}$&74,4$^{\circ}$&73,0$^{\circ}$\\
 $\gamma$&106,7$^{\circ}$&112,9$^{\circ}$&109,8$^{\circ}$\\
 $\delta$&67,2$^{\circ}$&73,1$^{\circ}$&70,2$^{\circ}$\\
\end{tabular}
\end{table}
\end{frame}

\begin{frame}
\frametitle{LEED des bedampften Re-Kristalls}
\begin{minipage}{\linewidth}
%\vspace{-0.4cm}
\begin{figure}[H]
		\captionsetup{name=Abb.}
	\begin{minipage}[b]{0.3\textwidth} 
		\includegraphics[width=\textwidth]{bilder/unbedampft_E207}
		\caption*{\textit{Re-Oberfläche}}
	\end{minipage}
	\hfill
	\begin{minipage}[b]{0.3\textwidth}
		\includegraphics[width=\textwidth]{bilder/0_5ML_E208}
		\caption*{\textit{1/2 Monolage Au}} 
	\end{minipage}
	\hfill
	\begin{minipage}[b]{0.3\textwidth} 
		\includegraphics[width=\textwidth]{bilder/1ML_E207}
		\caption*{\textit{1 Monolage Au}} 
	\end{minipage}
	
	\begin{minipage}[b]{0.3\textwidth}
		\includegraphics[width=\textwidth]{bilder/6ML_E207}
		\caption*{\textit{6 Monolagen Au}}
	\end{minipage}
	\hfill
	\begin{minipage}[b]{0.3\textwidth} 
		\includegraphics[width=\textwidth]{bilder/10ML_E207}
		\caption*{\textit{10 Monolagen Au}}
	\end{minipage}
	\hfill
	\begin{minipage}[b]{0.3\textwidth}
		\includegraphics[width=\textwidth]{bilder/30ML_E208}
		\caption*{\textit{30 Monolagen Au}}
	\end{minipage}
% 	\caption*{LEED-Aufnahmen vom Re-Kristall ohne
% 	und mit verschiedenen Bedeckungsgraden von Gold bei einer Elektronenenergie von \SI{208}{eV}.}
\end{figure}
\end{minipage}
\end{frame}

%___________________________________________________________________________________________________________________________________________
%___________________________________________________________________________________________________________________________________________


\subsection[Ergebnisse]{STM}

\begin{frame}
\frametitle{STM des Re-Kristalls}
\begin{figure}[htbp]
	\vspace{-0.5cm}
	\begin{minipage}[b]{0.45\textwidth} 
		\sffamily
		\includesvg[svgpath=bilder/]{rekristall1}
	\end{minipage}
	\hspace{0.5cm}
	\begin{minipage}[b]{0.45\textwidth}
		\sffamily
		\includesvg[svgpath=bilder/]{rekristall2}
	\end{minipage}
	\caption*{Regelmäßig angeordnete Terrassen, Breite etwa \SI{35}{nm}}
	\end{figure}
\end{frame}

\begin{frame}
\frametitle{STM halbe Monolage Au}
\begin{figure}
\begin{minipage}[b]{0.45\textwidth} 
		\includegraphics{bilder/profilhalbeMLkleiner}
	\end{minipage}
	\begin{minipage}[b]{0.45\textwidth}
		\includegraphics{bilder/halbeMLkleiner}
	\end{minipage}
% 	\caption{\textit{STM-Bilder von 0,5 Monolagen Gold auf Re. Es bilden sich Inseln aus
% 	mehreren Goldatomen von einer Größe von etwa \SI{7}{nm} Länge, die die Re-Oberfläche ungeordnet bedecken.}}
	\vfill
	\centering
	\vspace{0.3cm}\hspace{-1.5cm} 
	\includegraphics[height=3cm]{bilder/ProfilhalbeML.pdf}
% 	\caption{\textit{Höhenprofil aus Abb. \ref{halbeML2stm}a). Die Höhe der Goldinseln liegt bei etwa
% 	\SI{0,2}{nm}.}}
\end{figure}
\end{frame}

\begin{frame}
\frametitle{STM mehrere Monolagen Au}
\begin{figure}[H]
	\begin{minipage}[b]{0.45\textwidth} 
		\includesvg[svgpath=bilder/]{20ML}
	\end{minipage}
% 	\caption*{20 Monolagen Gold}
	\hspace{0.5cm} 
	\begin{minipage}[b]{0.45\textwidth}
		\includesvg[svgpath=bilder/]{30ML}
	\end{minipage}
% 	\caption*{30 Monolagen Gold}
\caption*{20 Monolagen Gold~~~~~~~~~~~~~~~~~~~~~~~~~~~~~~~~~~~~~~30 Monolagen Gold}
\end{figure}
\end{frame}

\begin{frame}
\frametitle{STM 20 Monolagen}
\vspace{-0.3cm}
\begin{figure}[H]
	\begin{minipage}[b]{0.45\textwidth} 
		\includegraphics{bilder/20ML_2kleiner}
		\caption*{20 ML ungetempert}
	\end{minipage}
	\hspace{0.5cm} 
	\begin{minipage}[b]{0.45\textwidth}
		\includegraphics{bilder/20MLget2kleiner}
		\caption*{20 ML \SI{10}{min} getempert bei \SI{2,0}{A}}
	\end{minipage}
% 	\caption{\textit{a) 20 Monolagen ungetempert, b) 20 Monolagen, 10 Minuten mit einem Filamentstrom
% 	von \SI{2,0}{A} getempert.
% 	Beim Tempern glättet sich die Oberfläche, es bilden sich größere, flache Inseln.}}
	\centering
		\includegraphics[height=3cm]{bilder/profiles20MLget}
% 	\caption{\textit{Höhenprofil aus Abb. \ref{getVergleich}b). Die Höhen der Oberflächen der Inseln
% 	schwanken im Bereich von etwa \SI{0,05}{nm}, somit sind die Inseln sehr eben.}}
\end{figure}
\end{frame}

\begin{frame}
\frametitle{Zusammenfassung}
%\framesubtitle{Subtitles are optional}
\begin{itemize}\setlength{\itemsep}{+15pt}
  \item Art des Wachstums nicht eindeutig, aber:
  \item 30 Monolagen als Molekülsubstrat geeignet
  \item gleiches Ergebnis für 20 Monolagen getempert: ca. \SI{30}{nm} bis \SI{35}{nm} im Durchmesser
  große, glatte Inseln
  \item periodische angeordnete Oberfläche (LEED)
  \item genügend große Flächen, um Anordnung von aufgebrachten organischen Molekülen zu untersuchen
\end{itemize}
\end{frame}


%___________________________________________________________________________________________________________________________________________
%___________________________________________________________________________________________________________________________________________
%___________________________________________________________________________________________________________________________________________


\section{Aufbau und Test einer Spraydepositionsapparatur}

%___________________________________________________________________________________________________________________________________________
%___________________________________________________________________________________________________________________________________________


\begin{frame}
\frametitle{}
%\framesubtitle{Subtitles are optional}
\begin{center}
\textcolor{darkred}{\huge{2. Teil:\\ \vspace{0.65cm}
 Aufbau und Test einer\\ \vspace{0.5cm}Spraydepositionsapparatur}}
\end{center}


\end{frame}

%___________________________________________________________________________________________________________________________________________
%___________________________________________________________________________________________________________________________________________


\subsection[Motivation]{Einleitung}

\begin{frame}
\frametitle{Wozu das Ganze?}
%\framesubtitle{Subtitles are optional}
\begin{itemize}\setlength{\itemsep}{+15pt}
  \item Funktioniert über ein Druckgefälle innerhalb der Apparatur
  \item Vorteil: kein thermisches Verdampfen $\Rightarrow$ keine Verformung der Moleküle
\end{itemize}
\end{frame}

%___________________________________________________________________________________________________________________________________________
%___________________________________________________________________________________________________________________________________________


\subsection[Versuchsaufbau]{Versuchsaufbau}

\begin{frame}
\frametitle{Schema der Spraydepositionsapparatur}
\begin{figure}[H]
\centering
\includegraphics{bilder/wuerfelklein.pdf}
\end{figure}
\end{frame}

\begin{frame}
\frametitle{Aufbau der Apparatur (1)}
%\framesubtitle{Subtitles are optional}
\begin{itemize}\setlength{\itemsep}{+15pt}
  \item Test an HV-Apparatur inkl. Vorpumpe $\&$ Hauptpumpe, Druckmessgerät (Vorteil kleineres
  Volumen)
  \item Substrat: Glasträger
  \item gelöste Moleküle: \\
  Kupfer-2-Phthalocyanin (CuPc) in Dimethylsulfoxid bzw. Dimethylformamid\\
  Tetracyanoquinodimethan (TCNQ) in Tetrahydrofuran
\end{itemize}
\end{frame}


\begin{frame}
\frametitle{Aufbau der Apparatur (2)}
\begin{figure}[H]
\centering
%\sffamily
\includesvg[svgpath=bilder/]{sb}
\end{figure}
\end{frame}

%___________________________________________________________________________________________________________________________________________
%___________________________________________________________________________________________________________________________________________


\subsection[Ergebnisse]{Test der Apparatur}

\begin{frame}
\frametitle{Erste Tests}
%\framesubtitle{Subtitles are optional}
\begin{itemize}\setlength{\itemsep}{+15pt}
  \item Getestete Zeiten: \\
  \SI{1}{ms} bis \SI{50}{ms} offenes Ventil ("`On-Time"'), dazwischen \SI{7}{ms} bis \SI{4}{s} geschlossenes
  Ventil ("`Off-Time"')\\ 
  gesamte Zyklusdauer bis 15 Minuten
  \item Druckanstieg in der Hauptkammer, aber nichts zu sehen auf Glasträger!
  \item Nach einigen Umbauten: Sichtbare farbige Spritzer auf Glasträger erst ohne Pumpe am Würfel
\end{itemize}
\end{frame}


\begin{frame}
\frametitle{Beobachtungen zwischendrin}
%\framesubtitle{Subtitles are optional}
\begin{itemize}\setlength{\itemsep}{+15pt}
  \item Ausgangsdruck in Hauptkammer \SI{e-5}{mbar} $\Rightarrow$ Druckanstieg bei Öffnen des
  Ventils auf \SI{e-3}{mbar}\\
  Druck in Würfel nur \SI{e-2}{mbar}
 % einigermaßen regulierbar über kurze On-, lange Off-Zeiten
  \item Öffnungen von Ventil $\&$ Konus anfällig für Verstopfungen $\Rightarrow$ kein Druckanstieg
  mehr in Hauptkammer
  \item Turbopumpe stark am Limit:\\
  Austausch der internen Vorpumpe durch externe Membranpumpe
\end{itemize}
\end{frame}

\begin{frame}
\frametitle{Test mit Quarzwaage}
%\framesubtitle{Subtitles are optional}
\begin{itemize}\setlength{\itemsep}{+15pt}
  \item 
  \item 
  \item 
\end{itemize}
\end{frame}

%___________________________________________________________________________________________________________________________________________
%___________________________________________________________________________________________________________________________________________
%___________________________________________________________________________________________________________________________________________



\section*{Summary}

\begin{frame}
\frametitle<presentation>{Summary}

\begin{itemize}
  \item The \alert{first main message} of your talk in one or two lines.
\end{itemize}




% The following outlook is optional.
\vskip0pt plus.5fill
\begin{itemize}
  \item Outlook
  \begin{itemize}
    \item Something you haven't solved.
    \item Something else you haven't solved.
  \end{itemize}
\end{itemize}
\end{frame}

\end{document}
